\documentclass[../main]{subfiles}

%解析力学のファイル

\begin{document}
    \section*{Introduction}

    \section{Lagrange形式}
    Newton力学に感銘をうけたLagrangeが始めた,より数学的に形式化された力学であり現代物理学の土台を成している.
    
    一般に高校で力学を学ぶとNewtonがほぼ全て現在の形式を作り上げたように思えるが,実はそうとも言い切れない.Philosophiæ Naturalis Principia Mathematicaで確かに微積分の概念は登場はしたものの,無限小の概念はまだ完成しておらずその完成はTaylorやCauchyなど様々な後世の研究者により行われた.
    
    Newtonの行った物理法則の証明は基本的には初等幾何的な手法を用いてのものであり,直感的な理解はしやすくも概念の拡張や他分野への応用など,少し抽象度の欠けるものであったことは否めなかった.

    力学を学びその万能感に酔いしれていると幾何的な直感を使えない対象が現れたときに足元を掬われてしまう.とはいえ大学で学ぶ力学ではあまり初等幾何的手法を用いない.どちらかといえば解析学的な方法(主に微分方程式)を用いて教わることがほとんどであるから,このようなことを言われてもあまりピンと来ない人がほとんどであろう.例えばこれを高校物理習いたての頃に言われれば少しは違うことを思うかもしれない.

    では,なぜこのようなものを学ぶ必要があるのかといえばこの先の物理を万分上ではNewtonの力学では手に余るからである.それは量子力学や統計物理学ももちろんそうだが,手近なところで言えば剛体や多体系などでも同じと言える.解析力学無しにこれらの対象を学ぼうとすれば困難を極めることは容易に想像がつくと思う.

    このことを踏まえるNewton力学もこれから学ぶ解析力学も単に形式の違いに過ぎないということがわかると思う.それゆえ,Newton力学が不要になったということではないことに注意してほしい.

    \subsection{d'Alembertの原理}

    \subsection{Lagrangianと作用積分}

    \subsection{最小作用の原理}

    \subsection{一般化座標}

    \subsection{変分原理}

    \subsection{Euler-Lagrange方程式}

    \subsection{Noetherの定理}

    \section{Hamilton形式}



    \section{場の解析力学}
\end{document}
